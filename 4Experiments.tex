
% !TeX root=DeepRelAttr.tex
% !TEX TS-program = pdfLatex

%%%%%%%%%%%%%%%%%%%%% EXPERIMENTS %%%%%%%%%%%%%%%%%%%%%%%%%%%%%%%%%%%%%
\section{Experiments}

\subsection{Datasets}
To assess the performance of the proposed method, we have evaluated our method on five datasets:
The first one is the \textbf{PubFig} \cite{pubfig} dataset (a set of public figure faces), which consists of 800 facial images of 8 random subjects, with 11 pre-defined attributes. The ordering of the attributes in images is annotated in a category level, \ie, all images in a specific category may be ranked higher, equal, or lower than all images in another category, with respect to an attribute.
\textbf{OSR} \cite{oliva2001modeling} dataset contains 2688 images in 8 categories, for which 6 relative attributes are defined. Like the PubFig dataset, relative ranking of attributes for this dataset is annotated in a category level. These two datasets are widely used in the literature, and therefore can provide us a good testbed to compare our results with the previous results acquired by the state-of-the-art methods. 
The third dataset is the  \textbf{UT-Zap50K} \cite{Yu2014} dataset (a collection of shoe images). This dataset consists of two collections, namely UT-Zap50K-1, in which \textit{coarse} relative attributes are compared for image pairs, and UT-Zap50K-2, in which \textit{fine-grained} relative attributes are compared for image pairs.  
Next, we have also conducted experiments on the \textbf{LFW-10} \cite{Sandeep_2014_CVPR} dataset. This dataset has 2000 images and 10 attributes. For each attribute, a random subset of 500 pairs of images have been annotated \hl{for each train and test set (what does this mean??)}. These two latter datasets have large number of categories, as well as large inter-sample varieties in terms of poses, lighting condition. This makes them quite challenging compared to PubFig and OSR.
In addition to these datasets, we use the \textbf{MNIST} \cite{lecun1998mnist} dataset, to further analyze the properties of our proposed end-to-end model and the obtained feature hierarchy. We have incorporated the class labels of the images as the relative attributes, and used the label values to rank images.
\subsection{Experimental setup}

Each attribute separately, because of datasets.
\subsection{Baseline and compared methods}
\subsection{Results}
\subsection{Discussions}

